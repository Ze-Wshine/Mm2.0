\documentclass[UTF8, zihao=-4]{ctexart}
\usepackage{minted}
\usemintedstyle{vs}
\usepackage{graphicx}
\usepackage{geometry}
\usepackage{atbegshi}					 % 用于绝对定位(导入图片)
\usepackage{fancyhdr}
\usepackage{fontspec}                    % 字体管理
\usepackage{titlesec}                    % 标题格式定制
\usepackage{setspace}                    % 行距控制
\usepackage{booktabs}
\usepackage{float}
\usepackage{amsmath}
\usepackage{caption}
\captionsetup[figure]{labelformat=empty}  % 取消图片前缀
\captionsetup[table]{labelformat=empty}   % 取消表格前缀
\usepackage{bm}  
\usepackage[backend=biber, style=numeric,sorting=none]{biblatex} % 管理参考文献
\usepackage{hyperref} % 添加超链接
\hypersetup{
	colorlinks=true,
	linkcolor=blue,    % 内部链接颜色
	citecolor=blue,     % 引用标记颜色
	urlcolor=blue       % URL颜色
}
\setminted{
	breaklines=true,    % 启用自动换行
	breakanywhere=true, % 允许在任意字符处换行(可选)
	breakautoindent=true % 保持换行后的缩进(可选)
}
\addbibresource{refs.bib}
%正文小四宋体 论文题目三号黑体   一级标题四号黑体居中   二三级标题小四黑体左对齐,单倍行距

\singlespacing
\setlength{\parskip}{0pt}


\title{
	\heiti\zihao{3} % 三号黑体
	\textbf{Title} \\ % 加粗
	\vspace{0.5em}
	
}

% 一级标题 (section)
\renewcommand{\thesection}{\chinese{section}} % 将 section 序号转为中文数字
\titleformat{\section}
{\centering\heiti\zihao{4}\bfseries} % 四号黑体加粗居中
{\thesection、}{0em}{}

% 二级标题 (subsection)
\renewcommand{\thesubsection}{\arabic{section}.\arabic{subsection}} % 强制使用阿拉伯数字
\titleformat{\subsection}
{\raggedright\heiti\zihao{-4}\bfseries} % 小四黑体加粗左对齐
{\thesubsection}{1em}{}

% 三级标题 (subsubsection)
\titleformat{\subsubsection}
{\raggedright\heiti\zihao{-4}\bfseries} % 小四黑体加粗左对齐
{\thesubsubsection}{1em}{}


% 设置正文字体页边距
\geometry{a4paper, left=3cm, right=2.5cm, top=2.5cm, bottom=2.5cm}
\pagestyle{fancy}  % 激活 fancyhdr 样式
\fancyhf{}         % 清空默认页眉页脚
\fancyfoot[C]{\thepage}  % 在页脚居中显示页码
\renewcommand{\headrulewidth}{0pt}  % 隐藏页眉

\begin{document}
	
	% 直接暴力插入图片作封面
	\newgeometry{margin=0cm, nohead, nofoot}
	\thispagestyle{empty} % 隐藏封面页页码
	
	
	\AtBeginShipoutNext{%
		\AtBeginShipoutUpperLeft{%
			\put(0,-\paperheight){
				\includegraphics[
				width=\paperwidth,
				height=\paperheight,
				keepaspectratio=false
%此处插入封面图片
				]{cover.png}%

			}%
		}%
	}
	\phantom{placeholder}
	\clearpage % 强制结束封面页
	\restoregeometry % 恢复正文字体边距
	
	\maketitle
	\thispagestyle{empty}
	
	% ==================== 自定义摘要环境 ====================
	\newenvironment{abstractwithkeywords}
	{
		\vspace*{2cm} % 顶部间距
		\centering
		{\heiti\zihao{3}\bfseries 摘\quad 要} \\ % 三号黑体加粗
		\vspace{1cm}
		\begin{minipage}{0.9\textwidth} % 控制摘要宽度
			\setlength{\parindent}{2em} % 恢复首行缩进
			\songti\zihao{-4} % 小四宋体正文
		}
		{ % 摘要结束后的关键词部分
		\end{minipage}
		\vspace{1cm}
		
		{\heiti\zihao{-4}\bfseries 关键词:} % 小四黑体加粗
		\heiti\zihao{-4}\bfseries 关键词 % 小四宋体
	}
	
	\begin{abstractwithkeywords} 
		摘要
	\end{abstractwithkeywords}
	
	% ---------- 正文 ----------
	\newpage
	\pagenumbering{arabic}  % 设置页码格式为阿拉伯数字
	\setcounter{page}{1}    % 重置页码为 1(封面页不计数)
	
	\section{问题重述}
	\subsection{问题背景}
	问题背景
	NIPT(Non-invasive Prenatal Test,无创产前检测)是一种通过采集孕妇外周血中胎儿的游离DNA片段,分析胎儿染色体是否异常的产前筛查技术。该技术主要用于早期发现如唐氏综合征(21号染色体异常)、爱德华氏综合征(18号染色体异常)和帕陶氏综合征(13号染色体异常)等常见染色体疾病。NIPT的准确性高度依赖于胎儿性染色体浓度:男胎Y染色体浓度需达到或超过4\%,女胎X染色体浓度需无异常。
	
	孕妇的孕周、BMI(身体质量指数)等因素会影响胎儿染色体浓度的检测准确性。因此,合理分组孕妇并确定最佳检测时点,对提高检测成功率、降低因延迟发现胎儿异常所带来的风险(如治疗窗口期缩短)具有重要意义。
	
	\subsection{问题要求} 
	问题一:分析胎儿Y染色体浓度与孕妇孕周数、BMI等指标的相关性,建立关系模型并检验其显著性。
	
	问题二:以男胎孕妇的BMI为主要因素,对其进行合理分组,确定每组的最佳NIPT时点,以最小化潜在风险,并分析检测误差对结果的影响。
	
	问题三:综合考虑体重、年龄、检测误差及Y染色体浓度达标比例等因素,对男胎孕妇的BMI进行分组,确定每组的最佳NIPT时点,以最小化潜在风险,并分析检测误差的影响。
	
	问题四:针对女胎孕妇,以21、18、13号染色体非整倍体为判定依据,结合X染色体Z值、GC含量、读段数、BMI等因素,建立女胎异常的判定方法。
	
	
	\section{符号说明}
	\begin{tabular}{@{}p{6cm} p{8cm}@{}}
		\toprule
		\textbf{符号} & \textbf{说明} \\
		\midrule
		$G_a$        &   孕周   		\\
		$V$          &  胎儿Y染色体浓度 \\
		
		\bottomrule
	\end{tabular}
	
	\section{问题分析}
	\begin{itemize}
		
		\item 针对问题一:本问题的核心是定量探究胎儿Y染色体浓度$V$、孕妇孕周数$G_a$和$BMI$值之间的统计关系。
		依题意,得有以下思路:
		
		首先进行\textbf{数据预处理},对孕周数据进行\textbf{标准化}处理,转换为连续数值(如12w+3d 转换为12.43周)。针对Y染色体浓度$V$,进行Logit变换或Arcsin平方根变换可以满足线性模型对因变量的要求。此外,还需处理缺失值,并基于测序质量指标(如GC含量、过滤比例)剔除异常样本。
		
		接着进行\textbf{探索性分析(EDA)},绘制Y染色体浓度随孕周变化的散点图和平滑曲线并按BMI分层展示其变化趋势。计算Spearman相关系数,以初步评估Y染色体浓度与孕周、BMI之间的相关性。
		
		然后开始\textbf{构建模型}:考虑到数据可能存在非线性和个体重复测量的情况(同一孕妇多次检测),我们决定在广义可加混合模型(GAMM) 和线性混合效应模型(LMM)做出选择。我们的模型将包含孕周$G_a$的非线性平滑项、BMI的线性(或平滑)项、以及可能的孕周与BMI的交互项。孕妇ID将被设置为随机截距,以控制个体差异。
		
		之后检验与解释\textbf{显著性},通常使用\textbf{似然比检验(LRT)},或者基于REML的\textbf{F检验}来评估孕周平滑项、BMI主效应及交互项的统计显著性。绘制不同BMI水平下Y染色体浓度随孕周变化的预测曲线来更直观解释变量间的关系。
		
		最后分析\textbf{稳健性}:检查残差、尝试不同的变量变换方式、在数据子集上进行交叉验证,以检验最终模型的稳健性和可靠性。
		
		
		
		\item 针对问题二:
		
		
		\item 针对问题三:
		
	\end{itemize}
		
	\section{问题一模型的建立与求解}
		\subsection{数据分析与处理}
		
		\subsection{建模求解}

	\section{问题二模型的建立和求解}
		\subsection{数据分析与处理}

		\subsection{建模求解}
	
	\section{问题三模型的建立与求解}
		\subsection{数据分析与处理}
	
	
		\subsection{建模求解}

	\subsection{模型效果检验}
	
	\subsection{策略分析}









	\defbibheading{bibliography}[\bibname]{%
		\section*{#1} % 不编号的 section
	}
	\renewcommand{\bibname}{参考文献} % 定义标题名称
%	\printbibliography
	
	\newpage
	
	\section*{\textbf{附录 A 支撑材料文件列表}}
	\begin{table}[H]
		\centering
		\begin{tabular}{c c c}
			\toprule
			\textbf{文件名} 		& \textbf{文件类型}  & \textbf{简介} \\
			\midrule
			附件.xlsx & Excel表格文件 & 孕妇数据集 \\
			
			\bottomrule
		\end{tabular}
	\end{table}
	
	\section*{\textbf{附录 B 源数据}}

	\section*{\textbf{附录 C 所用软件}}
	所用软件包括:Excel、PycharmCommunity、VisualStudioCode、TexStudio
	
	\section*{\textbf{附录 D 代码}}
	\inputminted[
	frame=single,
	framerule=1pt
	]{python}{Template.c}


\end{document}